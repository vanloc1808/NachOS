\section{Tổng quan}
\subsection{Giá trị của các thanh ghi}
\begin{itemize}
\item Mã system call được đưa vào thanh ghi r2.
\item Tham số thứ 1 được đưa vào thanh ghi r4.
\item Tham số thứ 2 được đưa vào thanh ghi r5.
\item Tham số thứ 3 được đưa vào thanh ghi r6.
\item Tham số thứ 4 được đưa vào thanh ghi r7.
\item Kết quả thực hiện của system call được đưa vào thanh ghi r2.
\end{itemize}
\subsection{Các bước viết một system call}
Muốn phục vụ người dùng có thể thực thi được các công việc khác nhau, người lập trình hệ điều hành phải xây dựng một bộ các system call đủ để phục vụ các yêu cầu này. System call cũng là hàm xử lý nhưng ở \textit{kernel mode}, khác với một hàm của chương trình người dùng ở \textit{user mode}. Sau đây là các bước viết một system call.\\
\textbf{Bước 1:} Trong tập tin \textit{/code/userprog/syscall.h
}, thêm dòng khai báo một syscall mới.\\
\textbf{Bước 2:} Thêm các dòng định nghĩa vào tập tin \textit{/code/test/start.c} và \textit{/code/test/start.s}.\\
\textbf{Bước 3:} Sửa điều kiện \textbf{if...} thành \textbf{switch... case} trong tập tin \textit{code/userprog/exception.cc}. Trong phần xử lý cho các syscall, tạo tập tin có sử dụng hàm \textbf{System2User()} và \textbf{User2System()} được hướng dẫn.\\
\textbf{Bước 4:} Viết chương trình ở mức người dùng để kiểm tra.\\
\textit{Bước 5:} Thêm đoạn code vào \textit{/code/test/Makefile}.\\
\textbf{Bước 6:} Biên dịch lại NachOS.\\
\textbf{Bước 7:} Thực thi chương trình test, nếu chương trình không báo lỗi thì xem như thành công.